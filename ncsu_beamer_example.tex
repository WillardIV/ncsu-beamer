\documentclass[aspectratio=169]{beamer}
% other options: sectionpages

\usetheme{ncsu}

\usepackage[english]{babel}
\usepackage{listings}
\usepackage{csquotes}
\usepackage[style=authoryear,backend=biber]{biblatex}
\usepackage{filecontents}
\usepackage{blindtext}
\usepackage{pgfpages}

%\setbeameroption{show notes on second screen}% you'll need pgfpages for this one and a suitable pdf viewer i.e. dspdfviewer
%\setbeameroption{show notes}

\lstset{%
  basicstyle=\scriptsize\ttfamily,
  backgroundcolor=\color{25Gray}
}

\addbibresource{bibliography.bib}

% The color palettes can be changed
%   primary: left color beam, Walfpack Red!
%   secondary: default block environment
%   tertiary: top header showing outline
%   structure: structure elements such as list items

\title{NC State Beamer Theme}
\subtitle{A theme for typesetting presentation slides using \LaTeX{}}
\author{Yixiao Chen}
\email{ychen269@ncsu.edu}
\institute{Department of Mathematics\\NC State University}
\date{\today}

\begin{document}

\maketitle

\section*{Outline}
\begin{frame}{Outline}
	\tableofcontents
\end{frame}

\section{Introduction}
\subsection{Main features}
\begin{frame}{Main features}
  \begin{itemize}
    \item supports all aspect ratios
    \item vector backgrounds ``drawn'' using Ti\emph{k}Z
    \item customizable colors
  \end{itemize}
  \bigskip
  Newest versions available unver \url{https://github.com/yixiaoX/ncsu-beamer}
\end{frame}

\note{This is how a note page looks like if you use the beameroptions ``show notes'' or ``show notes on second screen''}

\subsection{Colors}
\begin{frame}{Colors, 1/2}
  \begin{columns}
  \begin{column}{0.475\textwidth}
    All NC State Brand colors palette are predefined.
  \end{column}
  \begin{column}{0.475\textwidth}
  Core colors:\par
  \begin{tikzpicture}[%
      node distance=9em,
      every node/.append style={font=\scriptsize,
        minimum size=3ex}
  ]
    \node[rectangle,fill=WolfpackRed,label=east:Wolfpack Red] (1) {};
    \node[rectangle,fill=black, label=east:Wolfpack Black,right of=1] (2) {};
    \node[rectangle,fill=white, label=east:Wolfpack White,below of=1,node distance=4ex] (3) {};
  \end{tikzpicture}

  Extended colors:\par
  \begin{tikzpicture}[%
      node distance=9em,
      every node/.append style={font=\scriptsize,
        minimum size=3ex}
  ]
    \node[rectangle,fill=ReynoldsRed,  label=east:Reynolds Red]              (1) {};
    \node[rectangle,fill=PyromanFlame,   label=east:Pyroman Flame,   right of=1] (2) {};
    \node[rectangle,fill=HuntYellow,    label=east:Hunt Yellow,    below of=1,node distance=4ex] (3) {};
    \node[rectangle,fill=GenomicGreen,  label=east:Genomic Green,  right of=3] (4) {};
    \node[rectangle,fill=CarmichaelAqua,    label=east:Carmichael Aqua,    below of=3,node distance=4ex] (5) {};
    \node[rectangle,fill=InnovationBlue,    label=east:Innovation Blue,    right of=5] (6) {};
    \node[rectangle,fill=BioIndigo,     label=east:Bio-Indigo,     below of=5,node distance=4ex] (7) {};
  \end{tikzpicture}
  \end{column}
  \end{columns}
\end{frame}

\begin{frame}[containsverbatim]{Colors, 2/2}
  In addition to defining colors for single items, the color palettes of the theme can be changed.

  Try for example adding
  \begin{lstlisting}[%
    language={[LaTeX]TeX},
    texcsstyle=*\color{BioIndigo},
    moretexcs={setbeamercolor}
  ]
  \setbeamercolor{palette primary}{fg=white,bg=WolfpackRed}
  \setbeamercolor{palette secondary}{fg=white,bg=InnovationBlue}
  \setbeamercolor{palette tertiary}{fg=90Gray,bg=10Gray}
  \end{lstlisting}
  to your preamble.
\end{frame}

\section{Example environments}
\subsection{Figures and equations}
\begin{frame}{Figures and equations}
  \begin{columns}[onlytextwidth]
    \begin{column}{0.5\textwidth}
        \centering
        \begin{figure}
        \includegraphicscopyright[width=\textwidth]{photo.jpg}{Image courtesy of \href{http://openphoto.net/gallery/image/view/5468}{openphoto.net}}
        \end{figure}
    \end{column}
    \begin{column}{0.4\textwidth}
    Here is some regular text in a column. And there is an equation
    \begin{displaymath}
      f(x)=ax^2+bx+c
    \end{displaymath}
    Here is some \alert{important} text.
    \end{column}
    \end{columns}
\end{frame}

\subsection{Lists}
\begin{frame}{List environments}
  \begin{columns}[onlytextwidth]
    \begin{column}{0.5\textwidth}
      This slide has a list\dots
      \blinditemize[3]
      \vspace*{5mm}
      descriptions\dots
      \blinddescription[2]
    \end{column}
    \begin{column}{0.5\textwidth}
      as well as some enumerations
      \blindenumerate[4]
    \end{column}
    \end{columns}
\end{frame}

\subsection{Blocks}
\begin{frame}{Block environments}
    \begin{alertblock}{Note}
        This is important
    \end{alertblock}
    
    \begin{exampleblock}{Example}
        This is an example
    \end{exampleblock}

    \begin{theorem}[Pythagoras] 
        $ a^2 + b^2 = c^2$
    \end{theorem}
\end{frame}

\section{Further reading}
\begin{frame}{Further reading}
  \nocite{*}
  \printbibliography[heading=none]
\end{frame}

\end{document}
